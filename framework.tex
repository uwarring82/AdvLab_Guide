% ============================================================================
% INVARIANT FRAMEWORK
% Tier 0 — Constitutional Reference Document
% Version 0.1
% ============================================================================

\chapter{Invariant Framework}
\label{ch:framework}

\begin{tier0box}[title=Document Status]
\textbf{Version:} 0.1\\
\textbf{Status:} DRAFT\\
\textbf{Tier:} 0 (Constitutional)\\[0.3cm]
\textit{Harbour Tier 0 Framework (no parity implied with externally validated physical or legal laws)}
\end{tier0box}

% ----------------------------------------------------------------------------
\section{Purpose and Scope}
\label{sec:fw-purpose}

\subsection*{What this framework governs}

This framework establishes the non-negotiable boundary conditions for the conduct of the Advanced Physics Laboratory. It applies to all activities, decisions, and documents associated with the laboratory course.

\subsection*{Who it applies to}

All participants in the laboratory—organisers, tutors, and students—are bound by this framework for the duration of their involvement. Operational role definitions reside in Tier~2 documents and are causally bound to this framework by explicit naming.

\subsection*{What it does not govern}

This framework does not specify procedures, schedules, grading criteria, or pedagogical methods. It does not resolve individual disputes. It does not prescribe how laboratory work is conducted, only the conditions under which conduct and governance must operate.

% ----------------------------------------------------------------------------
\section{Core Principles}
\label{sec:fw-principles}

\begin{principlebox}[title=Principle 1 — Known Criteria]
No assessment, evaluation, or consequential decision may be made against criteria that were not accessible to the affected party before the relevant action or submission.
\end{principlebox}

\begin{principlebox}[title=Principle 2 — Proportional Authority]
Responsibility for outcomes may only be assigned to parties who possessed corresponding authority over those outcomes. No party may be held accountable for conditions outside their control.
\end{principlebox}

\begin{principlebox}[title=Principle 3 — Traceable Decisions]
Any decision affecting a participant's standing, progress, or evaluation must be traceable to an identifiable decision-maker and a stated basis. Decisions without attribution are void for enforcement purposes.
\end{principlebox}

\begin{principlebox}[title=Principle 4 — Separation of Layers]
Meaning (why), procedure (how), and governance (who decides) are distinct concerns. No document or decision may conflate these layers in a manner that obscures which layer is operative.
\end{principlebox}

\begin{principlebox}[title=Principle 5 — Symmetric Visibility]
Obligations imposed on any party must be visible to all parties. No binding expectation may be enforced if it was not accessible to those it binds and those it protects.
\end{principlebox}

\begin{principlebox}[title=Principle 6 — Correctable Error]
No single error, failure, or misjudgement—by any party—may result in irreversible exclusion from the laboratory, provided the error is acknowledged and a documented correction path exists.
\end{principlebox}

% ----------------------------------------------------------------------------
\section{Structural Analogies}
\label{sec:fw-analogies}

\begin{statusbox}
\textbf{Illustrative, Non-Binding}

The following analogies are offered for intuition. They carry no normative weight. If any analogy is removed, all principles remain in force unchanged.
\end{statusbox}

\begin{itemize}
    \item \textbf{Principle 1} is analogous to the requirement in metrology that measurement uncertainty be stated before a result is reported, not retrofitted afterward.
    
    \item \textbf{Principle 3} is analogous to the traceability chain in calibration: a measurement is only valid if it can be traced to a reference standard through documented steps.
    
    \item \textbf{Principle 4} is analogous to the separation between a physical model (theory), an experimental protocol (procedure), and the governance of a collaboration (authorship, data rights).
    
    \item \textbf{Principle 6} is analogous to error-correction in experimental design: a single anomalous data point does not invalidate an experiment if systematic review is possible.
\end{itemize}

% ----------------------------------------------------------------------------
\section{Boundary Statements}
\label{sec:fw-boundaries}

\subsection*{Relationship to downstream documents}

All documents derived from or associated with the Advanced Physics Laboratory—including essays, handbooks, companions, and operational guides—must remain compatible with this framework. No downstream document may contradict, override, or reinterpret the principles stated here.

\subsection*{What this framework does not do}

This framework does not derive operational rules. It does not specify deadlines, formats, or procedures. It does not assign roles or duties. These matters belong to handbooks and operational documents, which must comply with this framework but are not contained within it.

\subsection*{Intentional separation}

The separation of meaning, procedure, and governance is architectural. Documents addressing purpose and commitment (essays) are distinct from documents addressing operations (handbooks), which are distinct from this constitutional layer. This separation is deliberate and must be preserved.

% ----------------------------------------------------------------------------
\section{Amendment Discipline}
\label{sec:fw-amendment}

\subsection*{Proposal}

Any participant (organiser, tutor, or student) may propose an amendment to this framework in writing, stating the proposed change and its rationale.

\subsection*{Review}

Proposed amendments must be reviewed by at least two organisers and documented in a change log before adoption. Review must include:

\begin{itemize}
    \item Explicit confirmation that the proposed change does not introduce operational content, role-specific duties, or procedural rules.
    \item Guardian lock-compliance verification before adoption.
\end{itemize}

\subsection*{Adoption}

Amendments take effect only when recorded in the official version of the framework with a version number and date. No amendment applies retroactively.

\subsection*{Expectation of stability}

Amendments are expected to be rare. The framework is designed for longevity. Frequent amendment indicates architectural failure elsewhere in the document system.

\subsection*{Meta-amendment clause}

Changes to this Amendment Discipline section require the same review process as any other amendment, plus explicit acknowledgement that the amendment mechanism itself is being modified.

% ----------------------------------------------------------------------------
\section{Closing Statement}
\label{sec:fw-closing}

This framework is intended to remain valid across changes in personnel, curriculum, and operational practice. It serves as the shared reference point against which all other laboratory documents are measured. It does not inspire, instruct, or persuade. It defines what must hold.

% ----------------------------------------------------------------------------
\section*{Document Metadata}
\label{sec:fw-metadata}

\begin{center}
\begin{tabular}{@{}ll@{}}
\toprule
\textbf{Field} & \textbf{Value} \\
\midrule
Version & 0.1 \\
Status & DRAFT \\
Harbour Tier & 0 (Constitutional) \\
Created & 2026-01-15 \\
Last Amended & — \\
Amendment Log & See repository changelog \\
\bottomrule
\end{tabular}
\end{center}
