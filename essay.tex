% ============================================================================
% UNIFIED ESSAY
% Tier 1 — Shared Commitments
% Version 0.3
% ============================================================================

\chapter{Unified Essay}
\label{ch:essay}

\begin{tier1box}[title=Document Status]
\textbf{Version:} 0.3 — Draft\\
\textbf{Tier:} 1 (local designation: Sails \& Repair Kits)\\[0.3cm]
\textbf{Status: Aspirational, Non-Binding}\\[0.2cm]
This essay describes shared intentions and commitments.\\
It does not define enforceable rules or procedures.\\
Where action is required, the handbook governs.\\[0.3cm]
\textbf{Endorsement:} Local aspirational document.\\
No enforcement parity with Invariant Framework or Operational Handbooks.\\
Compliance with Invariant Framework v0.1.
\end{tier1box}

% ----------------------------------------------------------------------------
\section{Why This Lab Exists}
\label{sec:essay-why}

The advanced physics laboratory is not a service we provide or a test you pass. It is a space we build together—semester by semester, experiment by experiment, conversation by conversation.

What happens here matters. You will encounter real instruments, real uncertainties, and real limits—yours and ours. Some days will be frustrating. Some results will refuse to cooperate. This is not failure; this is physics.

We believe that learning to work carefully under uncertainty is one of the most valuable things a physics education can offer. But careful work cannot happen without trust. Trust that expectations are clear. Trust that effort is recognised. Trust that mistakes are part of the process, not the end of it.

This is why we hold ourselves to three commitments that are not slogans here:

\begin{description}[style=nextline]
    \item[Respect] for the work, the people, and the limits of what any of us can do.
    \item[Trust] earned through transparency, not assumed by role.
    \item[Responsibility] carried by everyone, proportional to what each of us can control.
\end{description}

These words mean nothing if they appear only in documents. They mean something only if they shape how we act when things go wrong.

% ----------------------------------------------------------------------------
\section{Commitments by Role}
\label{sec:essay-commitments}

The following sections describe what each group commits to uphold. These are not rules. They are intentions we make visible to one another—so that trust has somewhere to stand.

% - - - - - - - - - - - - - - - - - - - - - - - - - - - - - - - - - - - - - - -
\subsection{Organisers — Stewardship}
\label{sec:essay-organisers}

We are the organisers of this laboratory. Our role is to create and maintain the conditions under which serious work can happen.

\textbf{We commit to:}

\begin{itemize}
    \item Making expectations visible before they apply—not after.
    \item Ensuring that criteria for assessment are accessible to everyone they affect.
    \item Maintaining continuity across semesters, so that what you prepare for is what you encounter.
    \item Caring for the instruments, the spaces, and the people who use them.
    \item Acknowledging when something is our responsibility—and acting on it.
    \item Supporting tutors so they can support you.
\end{itemize}

We do not claim to be perfect. We do claim to be accountable.

% - - - - - - - - - - - - - - - - - - - - - - - - - - - - - - - - - - - - - - -
\subsection{Tutors — Mentorship}
\label{sec:essay-tutors}

We are the tutors of this laboratory. Our role is to accompany you through the work—not to do it for you, and not to leave you alone with it.

\textbf{We commit to:}

\begin{itemize}
    \item Being present and prepared when we are expected.
    \item Responding to your questions with honesty, even when the honest answer is ``I don't know.''
    \item Using our authority carefully—never arbitrarily.
    \item Giving feedback that helps you improve, not feedback designed to impress or discourage.
    \item Recognising the difference between a student who is struggling and a student who is not trying.
    \item Asking for help from organisers when something is beyond what we can handle.
\end{itemize}

We are not examiners. We are not obstacles. We are here because we once stood where you stand now.

% - - - - - - - - - - - - - - - - - - - - - - - - - - - - - - - - - - - - - - -
\subsection{Students — Participation}
\label{sec:essay-students}

We are the students of this laboratory. Our role is to engage with the work seriously—not because we are told to, but because serious engagement is the only way any of this becomes valuable.

\textbf{We commit to:}

\begin{itemize}
    \item Arriving prepared—having read what we were asked to read, having thought about what we were asked to think about.
    \item Taking responsibility for our own work, including our mistakes.
    \item Respecting the shared resources of the laboratory: instruments, time, and the attention of others.
    \item Engaging with feedback—even when it is difficult to hear.
    \item Asking questions when we do not understand, rather than pretending we do.
    \item Recognising that tutors and organisers are people, not services.
\end{itemize}

We are not here to perform compliance. We are here to learn how to do physics.

% ----------------------------------------------------------------------------
\section{Mutual Visibility}
\label{sec:essay-visibility}

Every section of this essay is visible to every role.

Organisers can read what students commit to. Students can read what tutors commit to. Tutors can read what organisers commit to.

This visibility is deliberate. It exists so that:

\begin{itemize}
    \item No one is bound by expectations they cannot see.
    \item Trust can be verified, not merely claimed.
    \item Accountability flows in all directions, not just downward.
\end{itemize}

Visibility does not mean equal authority. Organisers carry responsibilities that students do not. Tutors make judgements that organisers delegate to them. These differences are real and necessary.

But visibility means that authority cannot hide. If a commitment is worth making, it is worth making where others can see it.

% ----------------------------------------------------------------------------
\section{How to Use This Essay}
\label{sec:essay-howto}

This essay is not a contract. You cannot appeal to it in a dispute. You cannot enforce it against anyone.

It exists to make shared intentions legible—so that when we act, we act with awareness of what others expect and what we have offered.

\begin{statusbox}
\textbf{If you need to know what to do:} consult the handbook (Part~III).\\
\textbf{If you need to know what cannot be violated:} consult the invariant framework (Part~I).\\
\textbf{If you want to understand why we do things this way:} you are in the right place.
\end{statusbox}

\bigskip

We hope you find the work worthwhile. We hope you find the people trustworthy. We hope, when you leave, you take something with you that lasts longer than a grade.

% ----------------------------------------------------------------------------
\section*{Document Metadata}
\label{sec:essay-metadata}

\begin{center}
\begin{tabular}{@{}ll@{}}
\toprule
\textbf{Field} & \textbf{Value} \\
\midrule
Version & 0.3 \\
Status & DRAFT \\
Tier & 1 (Aspirational / Non-Binding) \\
Created & 2026-01-15 \\
Amended & 2026-01-15 \\
Parent Document & Invariant Framework v0.1 \\
\bottomrule
\end{tabular}
\end{center}
