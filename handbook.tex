% ============================================================================
% UNIFIED HANDBOOK
% Tier 2 — Operational Procedures
% Version 0.2.2
% ============================================================================

\chapter{Unified Handbook}
\label{ch:handbook}

\begin{tier2box}[title=Document Status]
\textbf{Version:} 0.2.2 — Draft\\
\textbf{Tier:} 2 (local designation: Toolkit \& Navigation)\\[0.3cm]
\textbf{Status: Procedural, Enforceable}\\[0.2cm]
This handbook defines operational procedures and enforceable expectations.\\
It does not define principles (see Invariant Framework) or intentions (see Essay).\\
Where boundaries are in question, the Invariant Framework governs.\\[0.3cm]
\textbf{Endorsement:} Operational document with binding force within stated scope.\\
Compatibility with Invariant Framework v0.1: verification pending; required before adoption.\\
All sections visible to all roles; visibility does not imply equal obligation.
\end{tier2box}

% ============================================================================
\section{How to Use This Handbook}
\label{sec:hb-howto}

This handbook is divided by role. Each section is visible to everyone, but binding only on the role it addresses.

\begin{center}
\begin{tabular}{@{}lll@{}}
\toprule
\textbf{If you are a...} & \textbf{Your primary section is...} & \textbf{Also read...} \\
\midrule
Student & Section~\ref{sec:hb-students} & Sections~\ref{sec:hb-common},~\ref{sec:hb-escalation} \\
Tutor & Section~\ref{sec:hb-tutors} & Sections~\ref{sec:hb-common},~\ref{sec:hb-students},~\ref{sec:hb-escalation} \\
Organiser & Section~\ref{sec:hb-organisers} & All sections \\
\bottomrule
\end{tabular}
\end{center}

\subsection*{Why is everything visible?}

So that no one is bound by expectations they cannot see, and so that trust can be verified. A student can read what tutors are expected to do. A tutor can read what organisers commit to. This is not surveillance—it is transparency.

\subsection*{Key Terms}

\begin{center}
\begin{tabular}{@{}L{3.5cm}L{10cm}@{}}
\toprule
\textbf{Term} & \textbf{Meaning} \\
\midrule
Tier 0 & The Invariant Framework — principles that cannot be changed by this handbook \\
Tier 1 & The Unified Essay — aspirational commitments, not enforceable \\
Tier 2 & This handbook — procedures that can be enforced \\
Visibility & Everyone can read a section; only the addressed role is bound by it \\
Compatibility check & Verification that a document does not contradict the Invariant Framework \\
Guardian veto & The right to block changes that would violate clarity, ethics, or Invariant Framework principles \\
\bottomrule
\end{tabular}
\end{center}

\subsection*{What this handbook does:}
\begin{itemize}
    \item Defines procedures that can be enforced
    \item Specifies timelines, responsibilities, and escalation paths
    \item Provides templates and checklists for practical work
\end{itemize}

\subsection*{What this handbook does not do:}
\begin{itemize}
    \item Define the principles of the laboratory (see Invariant Framework)
    \item Explain why any of this matters (see Essay)
    \item Override formal university examination regulations (see Section~\ref{sec:hb-references})
    \item Contain personal or private information (see below)
\end{itemize}

\subsection*{Where personal and private information lives}

All personalised information—including contact details, lab schedules, group assignments, and grading records—is stored exclusively on \textbf{ILIAS}. This ensures that private information remains behind authentication and is not distributed in public documents.

\begin{center}
\begin{tabular}{@{}ll@{}}
\toprule
\textbf{Information type} & \textbf{Location} \\
\midrule
Procedures and templates & This handbook \\
Principles and boundaries & Invariant Framework \\
Intentions and commitments & Unified Essay \\
Experiment-specific instructions & \texttt{experiments/<topic>/} \\
Contact details, schedules, assignments & ILIAS (authenticated) \\
Grading records and criteria weights & ILIAS (authenticated) \\
Safety training modules and records & ILIAS (authenticated) \\
\bottomrule
\end{tabular}
\end{center}

% ============================================================================
\section{Common Procedures}
\label{sec:hb-common}

These procedures apply to all roles. They are minimal and operational.

\subsection{Safety Training Requirement}
\label{sec:hb-safety-req}

\begin{tier2box}[title=Universal Requirement]
\textbf{Everyone who works in the laboratory must hold valid safety training.}

This applies to students, tutors, and organisers without exception. Safety training certificates are valid for \textbf{12 months} from the date of completion. You are responsible for renewing your training before it expires.

If your safety training has expired, you may not enter the laboratory until it is renewed.
\end{tier2box}

\subsection{Documentation}

All consequential actions, decisions, and communications should be documented in a form that can be retrieved later. Verbal agreements are valid but should be confirmed in writing when they affect assessment, scheduling, or responsibilities.

\subsection{Timeliness}

Deadlines stated in this handbook or in experiment guides are binding unless explicitly modified in writing by an organiser. Requests for extensions must be made before the deadline, not after.

\subsection{Communication Channels}

Official communications use the channels specified for each course (typically ILIAS and university email). Information communicated only through informal channels (verbal, personal messaging) does not override official documentation.

\subsection{Good Faith}

All parties are expected to act in good faith. In this handbook, good faith means:

\begin{itemize}
    \item Communicating honestly about problems, limitations, and mistakes
    \item Interpreting ambiguous situations reasonably rather than exploiting loopholes
    \item Attempting one direct resolution step before escalating (where it is safe and appropriate to do so)
\end{itemize}

Good faith is demonstrated through behaviour, not declared as an intention.

% ============================================================================
\section{For Students}
\label{sec:hb-students}

This section defines your operational responsibilities and provides practical tools for laboratory work.

% ----------------------------------------------------------------------------
\subsection{The 10-Minute Start}

\begin{statusbox}
\textbf{If you have 10 minutes before your first lab:}
\begin{enumerate}
    \item Read Section~\ref{sec:hb-prep-checklist} (Preparation Checklist)
    \item Print the Quick Reference Card (Section~\ref{sec:hb-quickref})
\end{enumerate}
That's enough to survive Day~1. Come back for the rest when you need it.
\end{statusbox}

% ----------------------------------------------------------------------------
\subsection{Safety}
\label{sec:hb-safety}

Laboratory safety protects you, your colleagues, and the equipment that future students will use.

\subsubsection{General Safety Habits}

\textbf{Three habits that apply everywhere:}

\begin{enumerate}
    \item \textbf{Know before you touch.} If you don't understand a piece of equipment, ask before using it.
    \item \textbf{Report problems immediately.} If something breaks, sparks, leaks, or behaves unexpectedly, tell a tutor or organiser at once.
    \item \textbf{Never work alone with hazardous equipment.} Some experiments require a second person present.
\end{enumerate}

\subsubsection{Mandatory Training}

Before entering the laboratory, you must complete the required safety training modules on \textbf{ILIAS}. These typically include:

\begin{itemize}
    \item General laboratory safety
    \item Electrical safety
    \item Laser safety (for optical experiments)
    \item Radiation protection (for experiments with radioactive sources)
    \item Cryogenic safety (for low-temperature experiments)
\end{itemize}

\textbf{Safety training is valid for 12 months.} If your training expires during the semester, you must renew it before your next lab session.

\textbf{Your participation in experiments may be refused if training records are incomplete or expired.}

\subsubsection{Personal Protective Equipment}

\begin{center}
\begin{tabular}{@{}ll@{}}
\toprule
\textbf{Hazard} & \textbf{Protection} \\
\midrule
Lasers (Class 3B, 4) & Laser safety goggles matched to wavelength \\
Chemicals & Lab coat, gloves, safety glasses \\
Cryogenics & Cryogenic gloves, face shield, closed shoes \\
High voltage & Insulated tools, no jewellery, buddy system \\
Radioactive sources & Dosimeter, distance, shielding, time limits \\
\bottomrule
\end{tabular}
\end{center}

% ----------------------------------------------------------------------------
\subsection{Checklists}

\subsubsection{Preparation Checklist (Before Lab)}
\label{sec:hb-prep-checklist}

\textbf{Administrative preparation:}
\begin{checklist}
    \item Safety training valid and not expiring before lab date (check ILIAS)
    \item Lab schedule and room confirmed on ILIAS
    \item Experiment guide read (\texttt{experiments/<topic>/})
\end{checklist}

\textbf{Conceptual preparation:}
\begin{checklist}
    \item Theory understood well enough to explain the research question
    \item Method concepts understood well enough to explain the measurement approach
    \item \textbf{Ready to give a 15-minute presentation to your tutor} (see below)
\end{checklist}

\textbf{Practical preparation:}
\begin{checklist}
    \item Safety hazards identified; questions noted for tutor
    \item Notebook page prepared using the template
    \item Success/failure signs written
    \item Rule for invalid runs written
    \item Relevant equations and expected order of magnitude noted
\end{checklist}

\subsubsection{Pre-Lab Presentation (Before First Measurements)}
\label{sec:hb-prelab}

\begin{tier2box}[title=Required Before Technical Work]
Before you begin technical work on any experiment, you will give a short presentation to your tutor.

\textbf{Duration:} Approximately 15 minutes (10 minutes presentation, 5 minutes discussion)
\end{tier2box}

\textbf{Content:}

\begin{center}
\begin{tabular}{@{}L{4cm}L{9cm}@{}}
\toprule
\textbf{Element} & \textbf{What to cover} \\
\midrule
Research question & What physical relationship or quantity are you investigating? \\
Why it matters & What is the scientific context? \\
Method concept & How does the measurement technique work in principle? \\
Expected outcome & What result do you expect? What order of magnitude? \\
Key uncertainties & What are the main sources of uncertainty you anticipate? \\
\bottomrule
\end{tabular}
\end{center}

\textbf{What this is not:} Not a test of memorisation; not graded separately.

\textbf{What this is:} A chance to surface misunderstandings early—a dialogue, not a monologue.

If your tutor identifies significant gaps in understanding, they may ask you to revisit the material before proceeding. This is not a penalty; it is an investment in the quality of your work.

\subsubsection{Lab Work Checklist (During Experiment)}

\textbf{Setup checks:}
\begin{checklist}
    \item Equipment identified (model or lab label recorded)
    \item Connections secure
    \item Settings documented
    \item Room conditions noted
\end{checklist}

\textbf{Measurement checks:}
\begin{checklist}
    \item Signal within expected range?
    \item Time base or scale bar recorded?
    \item Unexpected drift investigated?
\end{checklist}

\textbf{Stop rules:}
\begin{itemize}
    \item If something looks wrong: \textbf{Stop. Write it down. Ask your tutor.}
    \item If you cannot explain what the next adjustment is supposed to change: \textbf{Stop. Think first.}
\end{itemize}

\subsubsection{Data Analysis Checklist (After Lab)}

\begin{checklist}
    \item Raw data backed up (separate from working copy)
    \item Statistical uncertainty calculated
    \item Systematic uncertainties estimated and listed
    \item Uncertainties propagated through calculations
    \item Result compared to expected value
    \item Discrepancies investigated (not ignored)
\end{checklist}

% ----------------------------------------------------------------------------
\subsection{Understanding Uncertainties}
\label{sec:hb-uncertainties}

Every measurement has uncertainty. Reporting a result without uncertainty is like giving directions without distances.

\subsubsection{Two Kinds of Uncertainty}

\begin{center}
\begin{tabular}{@{}L{2.5cm}L{5cm}L{5.5cm}@{}}
\toprule
\textbf{Type} & \textbf{What it is} & \textbf{How to reduce it} \\
\midrule
Statistical & Random scatter in repeated measurements & Take more measurements; average reduces scatter as $1/\sqrt{N}$ \\
Systematic & Consistent bias in one direction & Identify the source; calibrate; compare with independent method \\
\bottomrule
\end{tabular}
\end{center}

\subsubsection{Estimating Statistical Uncertainty}

For $N$ repeated measurements:

\begin{enumerate}
    \item Calculate the \textbf{mean}: $\bar{x} = \frac{1}{N} \sum_{i=1}^{N} x_i$
    \item Calculate the \textbf{standard deviation}: $s = \sqrt{\frac{\sum_{i=1}^{N}(x_i - \bar{x})^2}{N-1}}$
    \item Calculate the \textbf{standard error}: $\sigma_{\bar{x}} = \frac{s}{\sqrt{N}}$
\end{enumerate}

\textbf{Example:} Five measurements of a time interval: 4.2, 4.5, 4.3, 4.1, 4.4\,ms.
\begin{itemize}
    \item Mean: 4.3\,ms
    \item Standard deviation: $\approx 0.16$\,ms
    \item Standard error: $\approx 0.07$\,ms
    \item Report: $(4.30 \pm 0.07)$\,ms
\end{itemize}

\subsubsection{Combining Uncertainties}

For independent uncertainties:

\textbf{Addition/subtraction:}
\begin{equation}
\sigma_z = \sqrt{\sigma_x^2 + \sigma_y^2}
\end{equation}

\textbf{Multiplication/division:}
\begin{equation}
\frac{\sigma_z}{z} = \sqrt{\left(\frac{\sigma_x}{x}\right)^2 + \left(\frac{\sigma_y}{y}\right)^2}
\end{equation}

\subsubsection{Reporting Uncertainties}

Always state: the value, the uncertainty, the units, and what the uncertainty represents.

\textbf{Good:} $(4.30 \pm 0.07)$\,ms (standard error, $N = 5$)

\textbf{Bad:} 4.3\,ms

\textbf{Also bad:} $4.302847 \pm 0.0712654$\,ms (false precision)

% ----------------------------------------------------------------------------
\subsection{Quick Reference Card}
\label{sec:hb-quickref}

\begin{quickrefbox}
\textbf{BEFORE LAB}\\
$\checkmark$ Safety training valid (12-month expiry)\\
$\checkmark$ Experiment guide read\\
$\checkmark$ Success/failure signs written\\
$\checkmark$ Ready to present method concepts to tutor\\[0.3cm]
\textbf{DURING LAB}\\
$\checkmark$ Check for saturation, drift, loose cables\\
$\checkmark$ Record settings, timestamps, anomalies\\
$\checkmark$ Stop if you can't explain the next step\\
$\checkmark$ Ask tutor if uncertain\\[0.3cm]
\textbf{AFTER LAB}\\
$\checkmark$ Back up raw data before editing\\
$\checkmark$ Statistical + systematic uncertainties\\
$\checkmark$ Propagate through calculations\\
$\checkmark$ Label axes, show error bars\\[0.3cm]
\textbf{DATA EXCLUSION}\\
Point to a pre-written sign, not post-hoc judgement
\end{quickrefbox}

% ----------------------------------------------------------------------------
\subsection{Your Rights and Recourse}

As a student, you have the right to:

\begin{itemize}
    \item \textbf{Know the criteria} by which you will be assessed before you submit work
    \item \textbf{Receive feedback} within the timeframes specified in Section~\ref{sec:hb-tutor-feedback}
    \item \textbf{Escalate concerns} through the paths defined in Section~\ref{sec:hb-escalation}
    \item \textbf{Be treated with respect} even when your work requires correction
\end{itemize}

% ============================================================================
\section{For Tutors}
\label{sec:hb-tutors}

This section defines your operational responsibilities as a tutor. Students can read this section—that is intentional.

% ----------------------------------------------------------------------------
\subsection{Safety Training}

The safety training requirements in Sections~\ref{sec:hb-safety-req} and~\ref{sec:hb-safety} apply to you as well.

\textbf{You must hold valid safety training to supervise laboratory work.} Your training certificates expire after 12 months. Verify your training status on ILIAS before the semester begins.

If your safety training expires, you may not supervise laboratory sessions until it is renewed.

% ----------------------------------------------------------------------------
\subsection{Feedback and Grading}
\label{sec:hb-tutor-feedback}

\subsubsection{Feedback Timelines}

\begin{center}
\begin{tabular}{@{}ll@{}}
\toprule
\textbf{Deliverable} & \textbf{Feedback deadline} \\
\midrule
Pre-lab presentation & Immediate (before measurements begin) \\
Lab notebook check & Same session or next scheduled session \\
Written report & Within 14 calendar days of submission \\
Presentation & Verbal immediately; written within 7 days if requested \\
Resubmission & Within 10 calendar days \\
\bottomrule
\end{tabular}
\end{center}

If you cannot meet a deadline, notify the student and an organiser \textbf{before} the deadline passes.

\subsubsection{Pre-Lab Presentation Conduct}

When a student presents their conceptual preparation:

\begin{itemize}
    \item Listen for understanding of the research question and method
    \item Ask clarifying questions—this is a dialogue, not an examination
    \item Do not proceed to measurements if the student cannot explain what they are measuring and why
    \item Document any decision to delay measurements
\end{itemize}

\subsubsection{Grading Consistency}

\begin{itemize}
    \item Apply the same criteria to all students for the same experiment
    \item Consult an organiser before finalising uncertain grades
    \item Do not retroactively change grading criteria
    \item Store grades on ILIAS only
\end{itemize}

\textit{These requirements implement Invariant Framework Principle~1 (Known Criteria) and Principle~3 (Traceable Decisions).}

% ----------------------------------------------------------------------------
\subsection{Authority and Limits}

\subsubsection{What You May Do}

\begin{itemize}
    \item Assess student work according to published criteria
    \item Conduct pre-lab presentations and determine readiness to proceed
    \item Require students to repeat measurements or revise reports
    \item Report safety violations or academic integrity concerns
\end{itemize}

\subsubsection{What You May Not Do}

\begin{itemize}
    \item Invent new assessment criteria not communicated in advance
    \item Grant extensions beyond course policy
    \item Exclude a student from the laboratory
    \item Waive safety requirements
    \item Supervise if your own safety training has expired
\end{itemize}

\textit{These limits implement Invariant Framework Principle~2 (Proportional Authority).}

% ----------------------------------------------------------------------------
\subsection{When to Escalate}

Escalate to an organiser when:

\begin{itemize}
    \item A student disputes a grade and you cannot resolve it
    \item You observe a safety violation or potential academic integrity issue
    \item A student's behaviour is disruptive
    \item A student repeatedly fails the pre-lab presentation
\end{itemize}

\textbf{How to escalate:} Email the course organiser with a factual summary. Do not delay.

% ============================================================================
\section{For Organisers}
\label{sec:hb-organisers}

This section defines your operational responsibilities as an organiser. Tutors and students can read this section—that is intentional.

% ----------------------------------------------------------------------------
\subsection{Infrastructure and Continuity}

\subsubsection{Pre-Semester Responsibilities}

\begin{itemize}
    \item Ensure all experiment guides are current
    \item Verify safety training modules are available on ILIAS
    \item \textbf{Verify that all tutors hold valid safety training}
    \item Confirm equipment is functional and calibrated
    \item Brief tutors on any changes
    \item Publish assessment criteria before students begin
    \item Ensure ILIAS contains current contact details and schedules
\end{itemize}

\subsubsection{During-Semester Responsibilities}

\begin{itemize}
    \item Maintain equipment or clearly mark out-of-service items
    \item \textbf{Monitor tutor safety training expiry dates}
    \item Respond to escalations within 5 working days
    \item Keep ILIAS information current
\end{itemize}

\subsubsection{Documentation}

\begin{itemize}
    \item Keep records of significant decisions
    \item Document exceptions granted
    \item Maintain equipment issue logs
    \item \textbf{Maintain records of safety training status for all personnel}
\end{itemize}

\textit{These requirements implement Invariant Framework Principle~3 (Traceable Decisions).}

% ----------------------------------------------------------------------------
\subsection{Escalation Resolution}

When a matter is escalated:

\begin{enumerate}
    \item \textbf{Acknowledge} receipt within 2 working days
    \item \textbf{Gather information} from all relevant parties
    \item \textbf{Decide} based on documented criteria
    \item \textbf{Communicate} the decision in writing
    \item \textbf{Document} the decision and its basis
\end{enumerate}

Most escalations should be resolved within 10 working days.

% ----------------------------------------------------------------------------
\subsection{Amendment Authority}

Organisers may amend Sections~\ref{sec:hb-students},~\ref{sec:hb-tutors}, and~\ref{sec:hb-organisers} subject to:

\begin{itemize}
    \item Compatibility check required before adoption
    \item Guardian may veto on lock-compliance grounds
    \item Changes take effect only when published with updated version
    \item Students and tutors must be notified
\end{itemize}

Organisers may \textbf{not} amend Section~\ref{sec:hb-common} or Section~\ref{sec:hb-escalation} without full review.

% ============================================================================
\section{Escalation Paths}
\label{sec:hb-escalation}

This section defines how concerns move through the system. It is binding on all roles.

% ----------------------------------------------------------------------------
\subsection{Principles of Escalation}

\begin{itemize}
    \item Escalation is not punishment—it is a mechanism for resolution
    \item Escalation should be timely—do not let problems fester
    \item Escalation should be documented
    \item Escalation preserves relationships—the goal is resolution, not blame
\end{itemize}

% ----------------------------------------------------------------------------
\subsection{Escalation Paths by Issue Type}

\begin{center}
\begin{tabular}{@{}L{3cm}L{2.5cm}L{2.5cm}L{3.5cm}@{}}
\toprule
\textbf{Issue} & \textbf{First contact} & \textbf{If unresolved} & \textbf{Final authority} \\
\midrule
Feedback not received & Tutor & Organiser & Organiser decision \\
Grade dispute & Tutor & Organiser & Organiser (documented) \\
Equipment failure & Tutor & Organiser & Organiser resolution \\
Safety concern & Tutor/Organiser & Organiser & Safety officer \\
Tutor conduct & Organiser & Institution & Institutional authority \\
Student conduct & Tutor$\to$Organiser & Institution & Institutional authority \\
\bottomrule
\end{tabular}
\end{center}

% ----------------------------------------------------------------------------
\subsection{Response Commitments}

\begin{center}
\begin{tabular}{@{}lll@{}}
\toprule
\textbf{Escalation level} & \textbf{Acknowledgement} & \textbf{Resolution target} \\
\midrule
Student $\to$ Tutor & Same/next session & 7 days \\
Student $\to$ Organiser & 2 working days & 10 working days \\
Tutor $\to$ Organiser & 2 working days & 10 working days \\
\bottomrule
\end{tabular}
\end{center}

% ============================================================================
\section{Upward References}
\label{sec:hb-references}

This handbook does not define principles. It points upward to documents that do.

\begin{center}
\begin{tabular}{@{}ll@{}}
\toprule
\textbf{Topic} & \textbf{Reference} \\
\midrule
Principles and boundaries & Invariant Framework (Tier 0) \\
Intentions and commitments & Unified Essay (Tier 1) \\
Uncertainty fundamentals & GUM (JCGM 100:2008) \\
Error propagation & Taylor, \textit{Introduction to Error Analysis} \\
Fitting and statistics & Bevington \& Robinson \\
Physical constants & CODATA; Particle Data Group \\
\bottomrule
\end{tabular}
\end{center}

\begin{statusbox}
\textbf{Institutional override:} Formal examination regulations of the university and faculty (Prüfungsordnung) override this handbook where applicable. This handbook governs local practice within that institutional envelope.
\end{statusbox}

% ============================================================================
\section*{Document Metadata}

\begin{center}
\begin{tabular}{@{}ll@{}}
\toprule
\textbf{Field} & \textbf{Value} \\
\midrule
Version & 0.2.2 \\
Status & DRAFT \\
Tier & 2 (Operational / Enforceable) \\
Created & 2026-01-15 \\
Parent Documents & Invariant Framework v0.1, Unified Essay v0.3 \\
Compatibility Check & Pending; required before adoption \\
\bottomrule
\end{tabular}
\end{center}

\subsection*{Amendment Hierarchy}

Amendment authorities for this handbook are subordinate to the Invariant Framework amendment discipline. No section may be amended in ways that conflict with the Invariant Framework (Tier~0). Compatibility verification is required before adoption. Guardian may veto on lock-compliance grounds.
